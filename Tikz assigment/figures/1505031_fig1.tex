\begin{figure}[h!]
    \centering
    \begin{tikzpicture}[scale=0.3]


  \fill[background]   (-10.158,-8.9) rectangle (10.05,10.1);
   \fill[myCyan] (-8.25,-8.5) rectangle (-2.08,-2.33); 
  % (-1.8,-1.5)]
   \fill[myCyan,rotate around={-26:(-1.8,1.3)}] (-7.97,1.3)  rectangle (-1.8,7.47); 
    \fill[myPurple] (-.3,-6.7) rectangle (2.4,-4);  
     \fill[myPurple] (-1.8,-1.4) rectangle (0.9,1.3);  
   \fill[myOrange]   (4,-8.2) rectangle (9.6,-2.6);
     \fill[myOrange]   (0.9,1.3) rectangle (6.5,6.9);
 \draw[color=myRed,thick] (-1.8,1.3) -- (0.9,1.3);
\draw[color=myCyan,thick] (0.5,1.8) -- (1,1.8);
\draw[color=myCyan,thick] (0.5,1.8) -- (.5,1.3);
  \draw[color=myRed,thick] (0.9,1.3)-- (0.9,6.9);
   \draw[color=myRed,thick] (-1.8,1.3) --(0.9,6.9);
  \node[above,color=textColor] at (-5,-6.1) {\textbf{\large c\textsuperscript{2}}};
  \node[above,color=textColor] at (-1.2,-6.1) {\textbf{\small{=}}};
  \node[above,color=textColor] at (1,-6.1) {\textbf{\large {a\textsuperscript{2}}}};
   \node[above,color=textColor] at (3.2,-6.1)  {\textbf{\small {+}}};
  \node[above,color=textColor] at (6.7,-6.1) {\textbf{\large b\textsuperscript{2}}};
  \node[below,color=textColor] at (-.4,1.3) {\textbf{\large a}};
    \node[right,color=textColor] at (.9,3.8) {\textbf{\large b}};  
     \node[left,color=textColor] at (-.4,4.2) {\textbf{\large c}}; 
    \end{tikzpicture}
     \caption{Visual representation of the famous Pythagorean theorem.
}
\end{figure} 