% Copyright 2004 by Till Tantau <tantau@users.sourceforge.net>.
%
% In principle, this file can be redistributed and/or modified under
% the terms of the GNU Public License, version 2.
%
% However, this file is supposed to be a template to be modified
% for your own needs. For this reason, if you use this file as a
% template and not specifically distribute it as part of a another
% package/program, I grant the extra permission to freely copy and
% modify this file as you see fit and even to delete this copyright
% notice. 

\documentclass{beamer}
\usepackage{subfiles}
\usepackage{color}
\usepackage{tikz}

% There are many different themes available for Beamer. A comprehensive
% list with examples is given here:
% http://deic.uab.es/~iblanes/beamer_gallery/index_by_theme.html
% You can uncomment the themes below if you would like to use a different
% one:
%\usetheme{AnnArbor}
%\usetheme{Antibes}
%\usetheme{Bergen}
%\usetheme{Berkeley}
%\usetheme{Berlin}
%\usetheme{Boadilla}
%\usetheme{boxes}
%\usetheme{CambridgeUS}
%\usetheme{Copenhagen}
%\usetheme{Darmstadt}
%\usetheme{default}
%\usetheme{Frankfurt}
%\usetheme{Goettingen}
%\usetheme{Hannover}
%\usetheme{Ilmenau}
%\usetheme{JuanLesPins}
%\usetheme{Luebeck}
\usetheme{Madrid}
%\usetheme{Malmoe}
%\usetheme{Marburg}
%\usetheme{Montpellier}
%\usetheme{PaloAlto}
%\usetheme{Pittsburgh}
%\usetheme{Rochester}
%\usetheme{Singapore}
%\usetheme{Szeged}
%\usetheme{Warsaw}


\definecolor{background}{rgb}{0.84, 0.92, 0.95}
\definecolor{myCyan}{rgb}{0.11, 0.71, 0.86}
\definecolor{myPurple}{rgb}{0.64, 0.46, 0.66}
\definecolor{myLightPurple}{rgb}{0.835, 0.686, 0.894}
\definecolor{myOrange}{rgb}{0.96, 0.66, 0.59}
\definecolor{textColor}{rgb}{0.31,0.20,0.53}
\definecolor{lineColor}{rgb}{0.68,0.45,0.45}
\definecolor{myRed}{rgb}{0.964, 0.172, 0.172}
\definecolor{circleBorder}{rgb}{0.09,0.17,0.39}
\definecolor{diameterColor}{rgb}{0.22,0.29,0.42}
\definecolor{myLightBlue}{rgb}{0.76,0.84,0.95}
\definecolor{myGreen}{rgb}{0.098, 0.709, 0.090}

\title{The Traveling Salesman Problem}

% A subtitle is optional and this may be deleted
%\subtitle{Optional Subtitle}

\author{Zahin Wahab \and Mursalin Habib}
% - Give the names in the same order as the appear in the paper.
% - Use the \inst{?} command only if the authors have different
%   affiliation.

\institute[BUET] % (optional, but mostly needed)
{
  
  Department of Computer Science \& Engineering\\
  Bangladesh University of Engineering and Technology
}
% - Use the \inst command only if there are several affiliations.
% - Keep it simple, no one is interested in your street address.

\date{\today}
% - Either use conference name or its abbreviation.
% - Not really informative to the audience, more for people (including
%   yourself) who are reading the slides online

\subject{Computer Science}
% This is only inserted into the PDF information catalog. Can be left
% out. 

% If you have a file called "university-logo-filename.xxx", where xxx
% is a graphic format that can be processed by latex or pdflatex,
% resp., then you can add a logo as follows:

\pgfdeclareimage[height=0.5cm]{university-logo}{university-logo-filename}
\logo{\pgfuseimage{university-logo}}

% Delete this, if you do not want the table of contents to pop up at
% the beginning of each subsection:


% Let's get started
\begin{document}

\begin{frame}
  \titlepage
\end{frame}



% Section and subsections will appear in the presentation overview
% and table of contents.

\begin{frame}{Problem Statement}
  \setbeamercovered{dynamic}

    \begin{itemize}
        \item<1-> \textbf{Input} : A complete undirected graph with non-negative edge costs.
    
        \item<2-> \textbf{Output} : A minimum cost tour i.e. a cycle that visits all the vertices exactly once.
    
    \end{itemize} 
\end{frame}



% You can reveal the parts of a slide one at a time
% with the \pause command:

\subfile{Subfiles/Example.tex}

\subfile{Subfiles/Example2.tex}


\begin{frame}{\centering The Traveling Salesman Problem}
\begin{block}{Question of the Day} \vspace{3mm}
\centering How do we find an optimal tour?\newline
\end{block}
\end{frame}

\begin{frame}{The Brute Force Algorithm}
\setbeamercovered{dynamic}
The Brute Force approach:
    \begin{itemize}
    \item<2-> Look at all possible tours in the graph. 
 \item<3-> Compute their costs.
 \item<4-> Pick the minimum from them.
    \end{itemize} 
\end{frame}


\begin{frame}{Brute Force Algorithm Complexity}
\setbeamercovered{dynamic}
However,
    \begin{itemize}
    \item<2-> In a graph on $n$ vertices, there are $(n-1)!$ TSP tours.
 \item<3-> Computing the cost of a tour takes linear time.
 \item<4->\textbf{Brute-Force Algorithm running time:} $\text{\# of tours} \times \text{ cost of computing one tour}$
 $= (n-1)! \times O(n)
 =O(n!)$
    \end{itemize} 
\end{frame}


\begin{frame}{An Efficient Algorithm?}
\setbeamercovered{dynamic}
 \begin{block}{A Question We Should be Asking Everyday}
  \vspace{3mm} \centering Can we do better? \newline
    \end{block}
\end{frame}

\begin{frame}{An Efficient Algorithm?}
   We can. But a polynomial time algorithm doesn't seem likely.
    \begin{itemize}
     \item<2->  \textit{The Traveling Salesman Problem} has been studied since the 1950s.No amount of significant progress has been made so far.
    \item<3->  \textbf{Edmonds' Conjecture (1965):} there is no polynomial time algorithm for the \textit{Traveling Salesman Problem}. 
    \item<4-> Edmonds' Conjecture equivalent to \(P\neq NP\).
    \item<5-> \textit{The Traveling Salesman Problem} is \alert{NP-Complete}!
    \end{itemize} 
\end{frame}

\subfile{Subfiles/CopingWithNPC}

\begin{frame}{Hard to even approximate}
\setbeamercovered{dynamic}
There is a catch. \newline \newline
 \begin{theorem}
 Unless $P = NP$, there does not exist a polynomial time $\alpha$- approximation algorithm for the \textit{Traveling Salesman Problem}.
    \end{theorem}
\end{frame}

\begin{frame}{Coping with NP-Completeness}
\setbeamercovered{dynamic}
We can-
    \begin{itemize}
    \item<0>  Solve TSP exactly, but take a really long time for it.
 \item<0> Solve it only approximately, but do it fast.
 \item<1-> Solve it exactly, but for really special cases.
    \end{itemize} 
\end{frame}

\subfile{Subfiles/MetricTSP}

\begin{frame}{Approximation Algorithms for Metric TSP}
\setbeamercovered{dynamic}

       \begin{itemize}
    \item Still NP-Complete!
 \item But there are good approximation algorithms.
 \begin{itemize}
     \item \textbf{The MST Heuristic} (a 2-approximation algorithm)
     \item \textbf{Christofides's Algorithm (1976)} (a $\frac{3}{2}$-approximation algorithm)
 \end{itemize}
 \end{itemize}
\end{frame}

% Placing a * after \section means it will not show in the
% outline or table of contents.

\begin{frame}{To Summarize}
  \begin{itemize}
  \item
    The \textit{Traveling Salesman Problem} is interesting.
  \item
    \textit{The Traveling Salesman Problem} is \alert{hard}! 
  \item
    Approximation algorithms for NP-Complete Problems are still an active area of research.
  \end{itemize}
\end{frame}



% All of the following is optional and typically not needed. 
\appendix


\begin{frame}[allowframebreaks]
  \frametitle<presentation>{Acknowledgements}
    
  \begin{thebibliography}{10}
    
  \beamertemplatebookbibitems
  % Start with overview books.

  \bibitem{Tim Roughgarden2016}
    Tim Roughgarden.
    \newblock {\em Stanford CS261 Lecture Notes}.
    \newblock 2016.
 
    
  \beamertemplatearticlebibitems
  % Followed by interesting articles. Keep the list short. 

  \bibitem{Jack Edmonds2000}
    Jack Edmonds.
    \newblock Paths, trees, and flowers.
    \newblock {\em Can. J. Math. 17: 449–467,
    1965.}
  \end{thebibliography}
\end{frame}

\end{document}


